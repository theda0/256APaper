\documentclass{article}

\usepackage[english]{babel}
\usepackage{amsmath}
\usepackage{amssymb}
\usepackage{amsthm}
\usepackage[letterpaper,top=2cm,bottom=2cm,left=3cm,right=3cm,marginparwidth=1.75cm]{geometry}
\usepackage{graphicx}
\usepackage[colorlinks=true, allcolors=blue]{hyperref}
\usepackage{fancyhdr}
\usepackage{tikz}
\usepackage{tikz-cd}
\usepackage{quiver}
\usetikzlibrary{matrix}
\usepackage[most]{tcolorbox}
\usepackage{hyperref}
\usepackage{array}
\usepackage{colonequals}
\usepackage{todonotes}
\usepackage{theoremref}

\font\maljapanese=dmjhira at 2.5ex
\newcommand{\yo}{\textrm{\!\maljapanese\char"48}}

\newtheorem{theorem}{Theorem}[section]

\theoremstyle{definition}
\newtheorem{lemma}[theorem]{Lemma}
\newtheorem{corollary}[theorem]{Corollary}
\newtheorem{proposition}[theorem]{Proposition}
\newtheorem{definition}[theorem]{Definition}
\newtheorem{example}[theorem]{Example}
\newtheorem{examples}[theorem]{Examples}


\theoremstyle{remark}
\newtheorem*{remark}{Remark}

\newcommand{\R}{\mathbb{R}}
\newcommand{\C}{\mathbb{C}}
\newcommand{\Z}{\mathbb{Z}}
\newcommand{\N}{\mathbb{N}}
\newcommand{\Q}{\mathbb{Q}}
\newcommand{\mb}[1]{\mathbb{#1}}
\newcommand{\mc}[1]{\mathcal{#1}}
\newcommand{\mk}[1]{\mathfrak{#1}}
\newcommand{\un}{\cup}
\newcommand{\ic}{\cap}
\pagestyle{fancy}
\newcommand\size{1}% distance of nodes from center

\usepackage{microtype}
\begin{document}


\title{The \'Etale Fundamental Group}
\author{Daniel Ao}

\maketitle

\tableofcontents

\section{Introduction}

The fundamental group is an important invariant that appears in algebraic topology and it is a natural question to ask if there is an analog in the algebro-geometric setting.
Every scheme has an associated topological space, so one may naively apply the definitions from algebraic topology; however, the topology on schemes is generally too coarse for us to obtain anything of value.
A more fruitful approach instead uses the fundamental group's relation to theory of covers, of which there is a satisfactory schematic analog in finite \'etale covers which we lay out in Section 3.\\

What may not be immediately obvious is that the \'etale fundamental group is intimately related to that of Galois theory for field extensions. 
In fact, as an easy corollary at the end of Section 3 we will prove \thref{grothendiecksformulation}, a (albeit slightly different) formulation of Krull's theorem for Galois extensions.
Before, this however, we will establish some necessary foundation in Galois Theory and the topological fundamental group. 
And although we may initially have cautioned against a relation between the topological and \'etale fundamental group, there are still contexts where one can say a lot about the other.
Near the end, we will look at one possibility in \thref{RiemannExistence}, which relates the two in the case of finite type schemes over $\C$ and see some immediate uses.\\

\indent We will roughly follow \cite{Szamuely}, in particular during our construction of the \'etale fundamental group we will try and emphasize the similarity to the topological case.
One may find many similar notions between Galois theory and both the \'etale and topological cases.
These are adequately addressed in a more axiomatic treatment involving Galois categories, for which one may look at Grothendieck's original work (\cite{grothendieck}) or at \cite{Lenstra}.

\section{Preliminaries}
We will first develop the important notion of a profinite group, and extend our knowledge of Galois theory for finite extensions to the possibly infinite case.

\subsection{Infinite Galois Theory}

Following the notation of \cite{Szamuely} we will denote a field extension $K \supseteq k$ as $K|k$. 
Now for finite Galois extensions $K|k$, there is an inclusion-reversing bijection between subfields $K \supseteq L \supseteq k$ and subgroups $H \subset \text{Gal}(K|k)$.
In particular, a subfield $L$ is assigned the subgroup $\text{Aut}(K|L) \subset \text{Gal}(K|k)$ fixing $L$ and a subgroup $H$ is assigned its fixed field $K^H$.
Unfortunately, for infinite Galois extensions this correspondence breaks down.
However, there is still a satisfactory correspondence which we will look at.\\

First note that any partially ordered set $(\Lambda, \leq)$ can be viewed as a category where the objects are elements of $\Lambda$, and there is a morphism between $a,b \in \Lambda$ if and only if $a \leq b$.
Call a partially ordered set $(\Lambda, \leq)$ \textit{directed} if for all $c_1, c_2 \in \Lambda$ there exists a $d \in \Lambda$ such that $c_1 \leq d, c_2 \leq d$.

\begin{definition}
	Let $\mc{C}$ be a category and $J$ a directed partially ordered set. A \textit{(filtered) inverse system} of $\mc{C}$ is a contravariant functor $F: J^{\text{op}} \to \mc{C}$.
\end{definition}

If the limit of an inverse system $F: J^{\text{op}} \to \mc{C}$ exists, we will call it an \textit{inverse limit}.
It will be denoted as $\varprojlim C_{j}$, where $\{C_j\}_{j \in J}$ are the objects in $\mc{C}$ indexed by $J$.
When $\mc{C}$ is the category of groups, the limit always exists and it is the subgroup of $\prod_{j \in J} G_j$ consisting of sequences $(g_i)$ such that $Fk(g_i) = g_j$ if there is a morphism $k: i  \to j$ in $\Lambda$.

\begin{definition}
	A profinite group is the inverse limit of an inverse system whose objects are all finite groups.
\end{definition}

\begin{example} \text{}
\begin{enumerate}
	\item For a group $G$, the set of finite quotients is naturally an inverse system.
		In particular, there is a partial order on the set of normal subgroups of finite index where $H_1 \leq H_2 \Longleftrightarrow H_1 \supseteq H_2$, and corresponding quotient maps between the associated groups $G/H_1 \to G/H_2$.
		The inverse limit of this system is called the \textit{profinite completion} of $G$ which we will denote as $\widehat{G}$.	
		The natural projection maps $G \to G/H_i$ commute with the system, and hence define a canonical map $G \to \widehat{G}$ from the property of limits.
	\item The Galois group for any field extension $K|k$ is a profinite group, as we will prove below.
\end{enumerate}
\end{example}
\begin{proposition}
	If $K|k$ is a Galois field extension then $\text{Gal}(K|k)$ is a profinite group.
	In particular, it is the inverse limit of an inverse system consisting of the groups $\text{Gal}(L|k)$ for finite Galois subextensions $K \supset L \supset k$.
\end{proposition}
\begin{proof}
	\todo{Fill in}
\end{proof}

	There is a natural topology on any profinite group which we will need to define the Galois correspondence for infinite extensions.
	If $G$ a profinite group which is the inverse system of finite groups $\{G_i\}_{i \in I}$ we may give each $G_j$ the discrete topology.
	These define a product topology on $\prod_{i \in I} G_i$, and so $G$ viewed as a subgroup can be equipped with the natural subspace topology.


\begin{theorem}[Krull]\thlabel{Krull}
	Let $K | k$ be a Galois extension.
	If $k \subseteq L \subseteq K$ is a subestension, then $\text{Gal}(K|L)$ is a closed subgroup of $\text{Gal}(K|k)$.
	In particular, for subextensions $L$ and closed subgroups $H \subset \text{Gal}(K|k)$ the maps $H \to K^H$ and $L \to \text{Gal}(K|L)$ form an inclusion reversing bijection.
\end{theorem}

\begin{proof}
	For a complete proof see \cite{Szamuely}, Theorem 1.3.11. 
\end{proof}

\begin{example}
	
\todo{Include an example}

\end{example}


\begin{remark}
	\todo{Remark about difference between finite case}
\end{remark}

We will rephrase \thref{Krull} in a more useful form by putting it in a categorical context, the next notion will be important in this regard, as well as in the construction of the \'etale fundamental group.

\begin{definition} \thlabel{etalealgebra}
	A finite-dimensional $k$-algebra $A$ is \textit{\'etale} (over $k$) if it is isomorphic to the finite direct product of separable extensions of $k$.
\end{definition}

\begin{proposition} \thlabel{etalealgebra}
	Let $A$ be a finite dimensional $k$-algebra.
	Then the following conditions are equivalent:
	\begin{enumerate}
		\item $A$ is \'etale
		\item $A \otimes_k \overline{k}$ is isomorphic to a finite direct product of copies of $\overline{k}$.
	\end{enumerate}
\end{proposition}
\begin{proof}
	The proof requires some algebraic machinery, see \cite{Szamuely}, Proposition 1.5.6.
\end{proof}

Now let $k^s \subset \overline{k}$ be the some separable closure of $k$ lying in an algebraically closed field $\overline{k}$.
We may view a finite separable extension $L|k$ as a $k$-algebra homomorphism $L \hookrightarrow \overline{k}$. Since any separable extension must lie in $k^s$, we may think of $k^s$ as the "largest" possible Galois extension.
In particular, finite separable extensions $L|k$ have a natural correspondence to the finite set $k$-algebra homomorphisms $\text{Hom}_k(L, k^s)$
 which defines a contravariant functor from the category of finite separable field extensions of $k$.
 Even more can be said though, the absolute Galois group $\text{Gal}(k^s|k)$ has a natural action on $k^s$, and thus define on action on each object $\text{Hom}_k(L, k^s)$, which for any $\lambda \in \text{Gal}(k^s|k)$ and $\phi \in \text{Hom}_k(L, k^s)$ acts by sending $\phi \to \lambda \circ \phi$.
 Hence, $\text{Hom}_k(_, k^s)$ is in fact a functor to the category of $\text{Gal}(k^s|k)$-sets.\\
\indent In fact, in lieu of \thref{etalealgebra} the functor $\text{Hom}_k(_,k^s)$ extends to the category of finite \'etale $k$-algebras, which gives us the following reformulation.

\begin{theorem}\thlabel{grothendiecksformulation}
	Let $k$ be a field. 
	Then the functor mapping a finite \'etale k-algebra $A$ to the finite set $\text{Hom}_k(A, k_s)$ gives an anti-equivalence between the category of finite sets with continuous (left) $\text{Gal}(k)$-action.
	Here separable field extensions give rise to sets with transitive $\text{Gal}(k^s|l)$-action and Galois extensions to $\text{Gal}(k^s|k)$ -sets isomorphic to finite quotients of $\text{Gal}(k^s|k)$.
\end{theorem}

\begin{proof}
	We will prove this later as a consequence of \thref{etalefunctor}. 
	However, an elementary proof is already in reach, a reader who wishes to view this may consult \cite{Szamuely}, Chapter 1.
\end{proof}

\subsection{The Topological Fundamental Group}
\todo{Fill in some exposition in this section, maybe add pictures}
We now begin a brief journey into topology and define the topological fundamental group.
From here on out we will refer to topological spaces as simple \textit{spaces} for brevity.


\begin{definition}
	Let $X$ be a topological space. 
	A \textit{path} in $X$ is a continuous map $s: [0,1] \to X$ from the unit interval.
	The two endpoints are $s(0)$ and $s(1)$, and if they coincide we may also call the path a \textit{loop}.
\end{definition}

We will call two paths $f,g: [0,1] \to X$ \textit{homotopic} if $f(0) = g(0), f(1), g(1)$ and there exists a continuous map $h: [0,1] \times [0,1] \to X$ such that $h(0,t) = f(t), h(1,t) = g(t), h(t,0) = f(0)$ and $g(t,1) = f(1)$ for all $t \in [0,1]$.
Intuitively, we may think of a homotopy between two paths $f$ and $g$ as a continuous deformation moving $f$ to $g$ in $X$, keeping the endpoints fixed. 
It is straightforward to check that homotopies define an equivalence relation on the set of paths in $X$.\\
\indent Additionally, for any two paths $f,g: [0,1] \to X$ where $f(1) = g(0)$ we can define their composition $f \circ g: [0,1] \to X$ where $(f \circ g )(t) = f(t)$ for $0 \leq t \leq \frac{1}{2}$ and $(f \circ g)(t) = g(2t - 1)$ for $ \frac{1}{2} < t \leq 1$ which we may think of as concatenating the two paths.
Of particular interest to us, any two paths starting and ending at the same point can be composed.

\begin{proposition}
	For a topological space $X$ and a choice of a basepoint $x \in X$, concatenation of loops which start and end at $x$ forms a group when restricted to homotopy classes.
\end{proposition}

This is a straightforward check. 
In particular, the identity is the homotopy class of the constant loop $s: [0,1] \to x \in X$ and the inverse of a homotopy class of the loop $s: [0,1] \to X$ is the homotopy class of the "backwards" path $s(1-t)$.

\begin{definition}[Fundamental Group]
	For a space $X$ with basepoint $x$, the (topological) \textit{fundamental group} $\pi^{\text{top}}_1(X,x)$	is the group of homotopy classes of loops with origin in $x$.
	The group operation is the same composition of loops defined previously.
\end{definition}

\begin{examples} \text{} \todo{clarify examples}
	\begin{enumerate}
		\item The circle $S^1$ has fundamental group $\Z$. 
		\item $\R^n$ has trivial fundamental group, such spaces are called \textit{simply connected}.
	\end{enumerate}
	
\end{examples}

\begin{definition}[Covering Spaces]
	A cover of $X$ consists of a topological space $Y$ and a map $p:Y \to X$ that satisfies the following property. 
	For all $x \in X$ there exists an open neighborhood $U$ of $x$ where $p^{-1}(U)$ decomposes into a nonempty disjoint union of open subsets $V_i$ of $X$ such that each map homeomorphically onto $U$ under $p$.
\end{definition}

Under the above conditions we may also call $Y$ a \textit{covering} of $X$.
We will be interested in the case when the fibers $p^{-1}(x)$ of a cover are finite for all $x \in X$ which we will call $Y$ a \textit{finite cover} or \textit{finite covering} of $X$.
Note that some authors require a covering space to be connected, we will not follow this convention and will explicitly state whenever $Y$ is connected.

\begin{example} Suppose that a topological group $G$ acts discretely on a space $Y$, that is, every $y \in Y$ has an open neighborhood $V$ such that the open sets $gV$ are pairwise disjoint for all $g  \in G$.
	Then the natural projection $Y \to G/Y$ under this action is a projection.
	For example, $\Z$ acts on $\R$ discretely by translation via the map $\Z \times \R \to \R$ sending $(n,r) \to n+r$.
	The corresponding cover $\R \to \R/\Z$ is a covering of $\R/\Z$ which is homeomorphic to $S^1$.
\end{example}

\indent A map between two covers $p_1: Y_1 \to X, p_2: Y_2 \to X$ is a continuous map $f: Y_1 \to Y_2$ making the below diagram commute.
% https://q.uiver.app/#q=WzAsMyxbMCwwLCJZXzEiXSxbMiwwLCJZXzIiXSxbMSwxLCJYIl0sWzAsMiwicF8xIl0sWzEsMiwicF8yIiwyXSxbMCwxLCJmIl1d
\[\begin{tikzcd}
	{Y_1} && {Y_2} \\
	& X
	\arrow["f", from=1-1, to=1-3]
	\arrow["{p_1}", from=1-1, to=2-2]
	\arrow["{p_2}"', from=1-3, to=2-2]
\end{tikzcd}\]
From the commutativity, we see that for any $x \in X$ the map $f$ restricts to map between the fibers $p_1^{-1}(x)$ and $p_2^{-1}(x)$.\\
\indent Now these maps along with the objects consisting of covers over $X$ defines a category.
Hence for each cover $p:Y \to X$ there is a group of automorphisms consisting of the maps $f: Y \to Y$ such that $f = f \circ p$.
We will denote this group as $\text{Aut}(Y|X)$ from now on, and call its elements \textit{deck transformations} following \cite{FomenkoFuchs}.
Now deck transformations must map fibers $p$ bijectively onto each other, which leads us to consider the following types of covers.

\begin{definition}[Regular Covers]
	A connected cover $p: Y \to X$ is \textit{regular} if the group of deck transformations acts transitively on each fiber.
\end{definition}

We will not immediately make use of this definition, but it will be handy to keep in mind.
The following lemma is very useful.

\begin{proposition}[Lifting Paths and Homotopies]	
	Let $p: Y \to X$ be a covering. 
	For any $\widetilde{x} \in Y$, and path $s: [0,1] \to X$ such that $s(0) = x = p(\widetilde{x})$, there exists a unique path $\widetilde{s}:[0,1] \to Y$ such that $\widetilde{s}(0) = \widetilde{x}$ and $p \circ \widetilde{s} = s$.
	In addition, if $s_1, s_2$ are two homotopic paths in $X$, their lifts $\widetilde{s_1}$ and $\widetilde{s_2}$ are homotopic in $Y$.
\end{proposition}

\begin{proof}
	See \cite{FomenkoFuchs}, Lecture 6.5.
\end{proof}
		
Consider a covering $p: Y \to X$ and let $x \in X$ be a chosen basepoint and $y$ an element of the fiber $p^{-1}(x)$.
	For any path $s$ representing a class $[\alpha] \in \pi_1(X,x)$ there is a unique lift $\widetilde{s}$ from the previous proposition.
	Define an action of $\pi_1(X,x)$ on the set $p^{-1}(x)$ by letting $[\alpha] \circ y = \widetilde{s}(1)$.
	Since $\widetilde{s}(1)$ is the same for homotopic paths, this action is well-defined by the last part of the previous proposition.
	We will call this (left) continuous action on $p^{-1}(x)$ the \textit{monodromy action}.\\
	\indent There is then a functor $\text{Fib}_x$, which takes a cover $Y \to X$ to the finite fiber $p^{-1}(x)$ of the basepoint.

	\begin{theorem}\thlabel{coverfunctor}
	For a space $X$ with base point $x$, the functor $\text{Fib}_x$ induces an equivalence between the category of finite covers of $X$ and the category of finite continuous left $\widehat{\pi_1(X,s)}$-sets.
	Connected covers correspond to finite $\widehat{\pi_1(X,s)}$-sets with transitive action and Galois covers correspond to space of open normal subgroups.
\end{theorem}
\begin{proof}
	See \cite{Szamuely}
\end{proof}

Compare with \thref{grothendiecksformulation}.
One may already conjecture the below relations
\begin{center}
\begin{tabular}{ |c c c| } 
\hline
	Absolute Galois Groups & $\Longleftrightarrow$ & $\widehat{\text{Fundamental Group}}$ \\
 
	Galois Extensions & $\Longleftrightarrow$ & Regular Coverings\\
 
	Separable Extensions & $\Longleftrightarrow$ & Connected Coverings\\
 \hline
\end{tabular}
\end{center}

One way to resolve this is with the notion of a \textit{Galois category} (see \cite{grothendieck} or \cite{Lenstra} for those interested in a detailed treatment), which both our category of finite \'etale $k$ algebras and finite covers are incarnations of. 
As one may speculate, any Galois category is equivalent to a category of finite $G$-sets for a profinite group $G$.
In our cases so far Galois extensions and regular coverings as well as separable extensions and connected coverings are respectively instances of \textit{Galois objects} and \textit{connected objects} in a Galois category.
We will not go over these treatments, but it may be useful to keep in mind the parallels especially as we begin develop another Galois category but in the context of schemes.


\section{The Algebraic Fundamental Group}

We can now start defining the scheme analogue of the fundamental group.
Setting up the correct notion of a finite cover for schemes requires a lot of care which we will start to go through.

\subsection{Finite \'Etale Covers}

We being with some preliminary definitions and start to familiarize ourselves with some important properties.

\begin{definition}[Finite morphism]
	An affine morphism $f: X \to Y$ of schemes is called \textit{finite} if for some affine open cover $\{U_i\}_{i \in I}  = \{\text{Spec}(A_i)\}_{i \in I}$ where $f^{-1}(U_i) = \text{Spec}(B_i)$, the natural ring homomorphism $A_i \to B_i$ exhibits $B_i$ as a finitely generated $A_i$-module.
\end{definition}

\begin{example}
	A closed immersion is finite as closed immersions are affine and the induced maps $\text{Spec}(A) \to \text{Spec}(A/I)$ are induced by surjective ring maps.
\end{example}

\begin{proposition}
	A finite morphism $f: X \to Y$ is proper. 
\end{proposition}
\begin{proof}
	We make use of the valuative criterion. \todo{cite criterion}
	Finite morphisms are automatically of finite type, and since they are affine they are also quasi-separated.
	Hence it suffices to show that for any valuation ring $R$ with field of fractions $K$ and commutative diagram
	% https://q.uiver.app/#q=WzAsNCxbMCwxLCJcXHRleHR7U3BlY30oUikiXSxbMCwwLCJcXHRleHR7U3BlY30oSykiXSxbMSwwLCJYIl0sWzEsMSwiWSJdLFsxLDJdLFswLDNdLFsyLDNdLFsxLDBdXQ==
\[\begin{tikzcd}
	{\text{Spec}(K)} & X \\
	{\text{Spec}(R)} & Y
	\arrow[from=1-1, to=1-2]
	\arrow[from=1-1, to=2-1]
	\arrow[from=1-2, to=2-2]
	\arrow[from=2-1, to=2-2]
\end{tikzcd}\]
that there exists a unique lift $\ell: \text{Spec}(R) \to X$ commuting with the diagram.
Denote $y \in Y$ the image of the generic point of $\text{Spec}(R)$, and consider an affine open $\text{Spec}(A)$ containing $y$.
It must also contain the closure of $y$ and hence the entire image of $\text{Spec}(R)$.
And letting $\text{Spec}(B) = f^{-1}(\text{Spec}(A))$, the above says that any lift $\ell$ must factor through the inclusion of $\text{Spec}(B)$.
In addition, any lift to $\text{Spec}(B)$ that commute with its map to $\text{Spec}(A)$ defines a lift of $f$, hence we may assume that $X$ and $Y$ are affine.
Then if $X = \text{Spec}(B'), Y = \text{Spec}(A')$, it suffices to show there is a unique ring homomorphism making the following diagram commute.
% https://q.uiver.app/#q=WzAsNCxbMCwxLCJSIl0sWzAsMCwiSyJdLFsxLDEsIkEnIl0sWzEsMCwiQiciXSxbMCwxLCIiLDEseyJzdHlsZSI6eyJ0YWlsIjp7Im5hbWUiOiJob29rIiwic2lkZSI6InRvcCJ9fX1dLFsyLDAsImMiXSxbMywxLCJkIiwyXSxbMiwzLCJmJyIsMl0sWzMsMCwiXFxwaGkiLDFdXQ==
\[\begin{tikzcd}
	K & {B'} \\
	R & {A'}
	\arrow["d"', from=1-2, to=1-1]
	\arrow["\phi"{description}, from=1-2, to=2-1]
	\arrow[hook, from=2-1, to=1-1]
	\arrow["{f'}"', from=2-2, to=1-2]
	\arrow["c", from=2-2, to=2-1]
\end{tikzcd}\]
Since $f'$ exhibits $B'$ as a finitely generated $A'$-module, every element of $B'$ is integral over $A'$.
However, since $R$ is a valuation ring it is closed in its field of fractions $K$, hence the image of $B'$ under $d$ must be contained in $R$.
This precisely says that $d$ factors through the inclusion $R \hookrightarrow K$, giving us our desired unique map.
\end{proof}

For a map $f: X \to Y$ of schemes the map $f^{\flat}: \mc{O}_Y \to f_*\mc{O}_X$ exhibits the direct image sheaf $f_* \mc{O}_X$ as an $\mc{O}_Y$-algebra.
In particular, $f_* \mc{O}_X$ is also an $\mc{O}_Y$-module.
This allows us to make the next definition.

\begin{definition}[Locally free morphism]	
	A finite morphism $f: X \to S$ is \textit{locally free} if the direct image sheaf $f_* \mc{O}_X$ is locally free of finite rank.   
\end{definition}

We can now define our main morphism of interest.
Immediately afterward we will record some basic properties we would expect.
\begin{definition}[Finite \'etale morphism]
	A locally free morphism $f$ where for each $s \in S$, the fiber $X_s$ is a finite \'etale $\kappa(s)$-algebra is called a \textit{finite \'etale morphism}.
	If $f$ is also surjective (on space) we may call $f$ a \textit{finite \'etale cover}.
\end{definition}


\begin{proposition}\thlabel{basechangecomp}
Finite \'etale morphisms are closed under base change and composition
\end{proposition}

\begin{proof}
	\todo{Prove or cite}
\end{proof}

\begin{proposition}\thlabel{etalediagonal}
	If $f: X \to S$ is an \'etale morphism, then ths diagonal morphism $X \to X \times_S X$ induced by $f$ is an isomorphism of $f$ onto an open and closed subscheme of $X \times_S X$.
\end{proposition}

\begin{proof}
	See \cite{Szamuely}, Proposition 5.2.7.
\end{proof}

\begin{proposition}
	Let $f\colon X \to S$ and $G: Y \to X$ be morphisms of schemes. 
	Then 
	\begin{enumerate}
		\item If $f \circ g$ is finite and $f$ is separated, then $g$ is finite.
		\item If in addition both $f \circ g$ and $f$ are finite \'etale, then so is $g$.
	\end{enumerate}
	
\end{proposition}

\begin{proof}	
	For the first statement, since $f$ is separated, it diagonal $\Delta_f$ is a closed immersion, and so also finite. 
	Since finite morphisms are closed under composition and base change, the Cancellation Lemma \todo{cite} gives us that $g$ must be finite as well.\\
\indent The second proposition follows similarly, since $f$ is finite \'etale \thref{etalediagonal} tells us that its diagonal $\Delta_f$ is an immersion onto on open and closed subscheme, which is also finite \'etale.
As \thref{basechangecomp} tells us that finite 'etale morphisms are closed under composition and basechange, the Cancellation Lemma tells us that $g$ is also finite \'etale.
\end{proof}

\begin{proposition}
	The image of a finite and locally free morphism $f: X \to Y$ (in particular a finite \'etale cover) is both open and closed.
\end{proposition}

\begin{proof}
	The image is closed because a finite morphism is proper and hence closed.
	To show that it is open, consider any point $y$ in the image of $f$, as $f_* \mc{O}_X$ is locally free as an $\mc{O}_Y$-module, there is an open neighborhood $U$ of $f$ such that 
	\[(f_* \mc{O}_X)|_U \cong (\mc{O}_Y|_U)^{\oplus n}\]
	But this says that $U$ must be contained in the image of $f$, hence its image must also be open.
\end{proof}

\begin{example}
	Trivial covers \todo{Fit example somewhere}
\end{example}

\begin{definition}[Geometric Point]
	Let $X$ be a scheme and consider a map $\overline{x}: \text{Spec}(k) \to X$ for $k$ a field.
	Denote $x \in X$ the image of the unique point of $\text{Spec}(k)$ under $\overline{x}$.
	We say $\overline{x}$ is a \textit{geometric point} if the field extension $\kappa(x) \supset k$ is a separable closure of $k$.
\end{definition}

\begin{definition}[Geometric Fiber]
	Let $S$ be a scheme. 
	Given a morphism $f: X \to S$ and a geometric point $\overline{x}: \text{Spec}(k) \to X$ the \textit{geometric fiber} $X_{\overline{x}}$ is the fiber product $\text{Spec}(k) \times_S X$ induced by $f$ and $\overline{x}$.
\end{definition}

\begin{remark}
As a result of \thref{etalealgebra} for finite dimensional \'etale algebras over a field, the geometric fiber of a finite \'etale morphism $f: X \to S$ for a geometric point $\overline{s} : \text{Spec}(\Omega) \to S $ is isomorphic to the finite disjoint union $\text{Spec}(\Omega \times \dots \times \Omega)$ of copies of $\text{Spec}(\Omega)$, which one may view as analogous to the finite fibers in the topological case.
\end{remark}

In pursuit of this connection, one may again for a map of schemes $f: X \to S$ denote $\text{Aut}(X|S)$ as the set of automorphisms of $X$ consisting of the maps $\lambda$ such that $f  = f \circ \lambda$.
This will indeed still be the correct notion, in the case where $f$ is finite \'etale consider the following diagram
% https://q.uiver.app/#q=WzAsNixbMSwxLCJYIl0sWzAsMSwiUyJdLFswLDAsIlxcdGV4dHtTcGVjfShcXE9tZWdhKSJdLFsxLDAsIlxcdGV4dHtTcGVjfShcXE9tZWdhKVxcdGltZXNfUyBYIl0sWzIsMSwiWCJdLFsyLDAsIlxcdGV4dHtTcGVjfShcXE9tZWdhKVxcdGltZXNfUyBYIl0sWzAsMV0sWzIsMSwiXFxvdmVybGluZXtzfSJdLFszLDJdLFszLDBdLFszLDEsIiIsMSx7InN0eWxlIjp7Im5hbWUiOiJjb3JuZXIifX1dLFs0LDAsIlxcbGFtYmRhIiwxXSxbNSwzXSxbNSw0XSxbNSwwLCIiLDEseyJzdHlsZSI6eyJuYW1lIjoiY29ybmVyIn19XV0=
\[\begin{tikzcd}
	{\text{Spec}(\Omega)} & {\text{Spec}(\Omega)\times_S X} & {\text{Spec}(\Omega)\times_S X} \\
	S & X & X
	\arrow["{\overline{s}}", from=1-1, to=2-1]
	\arrow[from=1-2, to=1-1]
	\arrow["\lrcorner"{anchor=center, pos=0.125, rotate=-90}, draw=none, from=1-2, to=2-1]
	\arrow[from=1-2, to=2-2]
	\arrow[from=1-3, to=1-2]
	\arrow["\lrcorner"{anchor=center, pos=0.125, rotate=-90}, draw=none, from=1-3, to=2-2]
	\arrow[from=1-3, to=2-3]
	\arrow[from=2-2, to=2-1]
	\arrow["\lambda"{description}, from=2-3, to=2-2]
\end{tikzcd}\]
By definition the leftmost square and the entire rectangle, but by gluing of pullbacks this means the righthand square is also a pullback.
As $\lambda$ is an isomorphism, so is the induced map on the geometric fiber $\text{Spec}(\Omega)\times_S X$, hence $\text{Aut}(X|S)$ has a natural action on geometric fibers for a given geometric point.

\begin{proposition}
	Let $S$ be a connected scheme, and $f\colon X \to S$ an affine surjective morphism.
	Then $f$ is a finite \'etale cover if and only if there is a finite, locally free and surjective morphism $g: Y \to S$ such that $X \times_S Y$ is a trivial cover of $Y$.
\end{proposition}
\todo{Check if prop is necessary}

\begin{lemma}
	Let $X, Z$ be two $S$-schemes where $Z$ is connected and $X$ finite \'etale (over $S$). 
	If $f_1, f_2: Z \to X$ are two $S$-scheme morphisms and $\overline{z}$ a geometric point of $Z$  such that $f_1 \circ \overline{z} = f_2 \circ \overline{z}$ then $f_1 = f_2$.
\end{lemma}

\begin{proof}
	\todo{Fill in}
\end{proof}

\begin{corollary}
	If $f: X \to S$ is a connected finite \'etale cover, then nontrivial elements of $\text{Aut}(X|S)$ act without any fixed points on any geometric fiber.
	In particular, $\text{Aut}(X|S)$ is finite.
\end{corollary}

\begin{proof}
	For any nontrivial automorphism $\lambda$ of $X$, apply the previous proposition where $f_1, f_2$ are $\text{id}_x$ and $\lambda$ respectively.
\end{proof}

With the notion of an action on the fibers of a cover, we have an analogous definition to that in the topological case.

\begin{definition}[Finite \'etale Galois cover]
	A connected finite \'etale cover $X \to S$ is called \textit{Galois} if $\text{Aut}(X|S)$ acts transitively on every geometric fiber.
\end{definition}


\subsection{The \'Etale Fundamental Group}


We are now in a similar situation as when we defined the topological fundamental group.
Let $\text{Fet}_S$ denote the category of finite \'etale covers of $S$.
Fix a geometric point $\overline{s}$ of $S$, there is a natural functor to the category of finite sets \textbf{Fin} by taking the geometric fiber over $\overline{s}$. 
The analogy of the monodromy action, on the other hand, is not immediately obvious.
Instead consider the following.

\begin{definition}[The \'Etale Fundamental Group]
	For a scheme $S$ and a geometric point $\widetilde{s}: \text{Spec}(K) \to S$, we define the \textit{\'etale fundamental group} $\pi_1(S, \widetilde{s})$ as the automorphism group of the fiber functor from $\text{Fet}_{S}$ to $\textbf{Fin}$.
\end{definition}

Here we are viewing the fiber functor as an object of the category consisting of functors from $\text{Fet}_S$ to $\textbf{Fin}$.


\begin{proposition}
	Let $f: X \to S$ be a connected finite \'etale cover. 
	There is a morphism $\pi: P \to X$ such that $f \circ \pi: P \to S$ is a finite \'etale Galois cover, and moreover every $S-$scheme morphism from a Galois cover to $X$ factors through $\pi$.
\end{proposition}

\begin{proof}
	\todo{Put in if space permits}
\end{proof}

\begin{definition}[Pro-representable functors]
	Let $C$ be a category and $F: C \to \text{Set}$ a set-valued functor.
	We will say that $F$ is \textit{pro-representable} if it is the limit of an inverse system in the category $\text{Set}^C$ of set valued-functors with indexing objects representable presheaves.
\end{definition} 

\todo{Fill in below proofs}
\begin{proposition}
	Let $X$ be a finite \'etale $S$-scheme and $\overline{s}$ a geometric point of $S$.
	Then the fiber functor $\text{Fib}_{\overline{s}}$ is pro-representable.
\end{proposition}

\begin{corollary}
	Every automorphism of the functor $\text{Fib}_{\overline{s}}$ comes from a unique automorphism of the inverse system constructed in the above proof.
\end{corollary}

\begin{corollary}
	The automorphism groups $\text{Aut}(X_i)^{\text{op}}$ form an inverse system whose limit is isomorphic to $\pi_1(S, \overline{s})$.
\end{corollary}

\begin{theorem}[Grothendieck]\thlabel{etalefunctor}
	Let $S$ be a connected scheme, and $\widetilde{s}: \text{Spec}(K) \to S$ a geometric point. Then \begin{enumerate} 
		\item The group $\pi_1(S, \overline{s})$ is profinite, and its action on $\text{Fib}_{\overline{s}}(X)$ is continuous for all $X$.
		\item The functor $\text{Fib}_{\overline{s}}$ induces an equivalence between $\text{Fet}_S$ and the category of finite continuous (left) $\pi_1(S, \overline{s})$-sets.	In addition, connected covers correspond to sets with transitive action, and Galois covers to finite quotients of $\pi_1(S, \overline{s})$.
 \end{enumerate} 
\end{theorem}

As a corollary we now give the promised proof of \thref{grothendiecksformulation}.
Let $S = \text{Spec}(k)$ for $k$ a field; a finite \'etale cover corresponds to the affine scheme of a finite dimensional \;etale $k$-algebra $L$.
For a geometric point $\text{Spec}(\Omega) \to S$, the corresponding geometric fiber is the pullback of affine schemes $\text{Spec}(\Omega \otimes_k L)$, which as a set it is indexed by all $k$-algebra homomorphisms from $L$ to $\Omega$ \todo{Elaborate with primitive element theorem}.
Since the image of every map must lie in the separable closure of $k$ within $\Omega$, we have an isomorphism 
\[\text{Fib}_{\overline{s}}(\text{Spec}(L)) \cong \text{Hom}_k(L, k^s)\]
and so we must have $\pi_1(S, \overline{s}) \cong \text{Gal}(k^s|k)$.



\section{The Riemann Existence Theorem}

As promised in the introduction, we will introduce the Riemann Existence Theorem and begin to explore some of its uses and consequences.
In order to transition from the \'etale context to the topological context, we will need a "nice enough" associated space to some scheme.\\

\indent If $X$ is a finite type scheme over $\C$ then for an affine open $U$ of $X$, $\Gamma(\mc{O}_X, U)$ is a finite type $\C$-algebra and so can be expressed of the form $\C[x_1, \dots, x_n]/(f_1, \dots, f_n)$ for polynomials $f_1, \dots, f_n$.
It follows $U$ is of the form $\text{Spec}(\C[x_1, \dots, x_n]/(f_1, \dots, f_n))$, of which the closed points correspond to the common zeros of $f_1, \dots, f_n$ in $\C^n$, which when viewed a subspace allows us to equip it with a natural analytic topology.
In the same way a cover of affine opens glue together to form $X$, so do their associated subspaces, giving us an associated analytic subspace $X^{\text{an}}$.
And as a map $X \to Y$ finite type schemes over $\C$ preserve closed points (the set of which is in fact a representable functor, \todo{cite}), it induces a corresponding map $X^{\text{an}} \to Y^{\text{an}}$.
This already allows us to state the Riemann Existence Theorem.


\begin{theorem}[Riemann Existence]\thlabel{RiemannExistence}\thlabel{riemannexistence}
	Let $X$ be a connected scheme of finite type over $\C$.
	Then the functor $(Y \to X) \mapsto (Y^{\text{an}} \to X^{\text{an}}))$ induces an equivalence between the categories of finite \'etale covers of $X$ and finite topological covers of $X^{\text{an}}$.
	As a result, for every $\C$ point $\overline{x}: \text{Spec}(\C) \to X$ this functor induces an isomorphism
	\[\widehat{\pi_1^{\text{top}}(X^{\text{an}}, \overline{x})} \xrightarrow{\sim} \pi_1(X, \overline{x})\]
\end{theorem}

\begin{proof}
	We will not go over the proof of the first statement, the more difficult part is in the essential surjectivity of the functor, this is overcome in $\cite{grothendieck}$ with the help of tools such as resolution of singularities.
	The relation between the two fundamental groups is an immediate application of \thref{coverfunctor}, \thref{etalefunctor}.
\end{proof}

Just working from the definition of the \'etale fundamental group, it is very difficult to figure out how to even begin computing such groups, even for simple schemes. 
As we saw at the end of the previous section that even for simple maps of field valued points given by field extensions, it is equivalent to computing their corresponding Galois groups!
However, \thref{riemannexistence} now introduces the ability to immediately begin computing certain \'etale fundamental groups using their topological analogue as we will highlight.

\begin{examples} \text{}
	\begin{enumerate} 	\item Consider $\textbf{A}_{\C}^1 - \{0\} \cong \text{Spec}(\C[x^{\pm 1}])$. 
		It's associated analytic space is simply $\C - \{0\}$, which has fundamental group $\Z$ for any choice of basepoint, so for any geometric point $\overline{x}$ of $\textbf{A}_{\C}^1 - \{0\}$ we conclude
		\[\pi_1(\textbf{A}_{\C}^1 - \{0\}, \overline{x}) \cong \widehat{\Z}\]
		This generalizes easily for $\textbf{A}_{\C}^n$  and localizations across more than one maximal ideal. 
		\item For $\textbf{P}_{\C}^n$, we have that its associated analytic space is just the topological complex projective space $\C P^n$.
	One may see this from the former description as gluing copies of $\textbf{A}_{\C}^n$ and the analogous set of charts on $\C P^n$.
	The topological theory gives us that 
	\[\pi_1^{\text{top}}(\C P^n, x) = 0\]
	so we conclude the same for $\pi_1(\textbf{P}_{\C}^n, \overline{x})$.
\item  \todo{Add complex surfaces of genus g }
\end{enumerate}
\end{examples}

And while we have only give examples for schemes over $\C$, the below result allows us to extend it to a slightly larger class of schemes. 
\begin{proposition}
	Let $K \supset k$ be an extension of algebraically closed fields, and let $X$ be a proper integral scheme over $k$.
	Then the natural map $\pi_1(X_L, \overline{x}_K) \to \pi_1(X, \overline{x})$ is an isomorphism for all geometric points $\overline{x}$ of $X$.
\end{proposition} 

\begin{proof}
	See \cite{Szamuely}, Proposition 5.6.7.
\end{proof}

\begin{remark}
	As a general application if results about the \'etale fundamental group holds for schemes over $\C$ they hold over any field  $k$ of characteristic zero.
	To do this one would apply the above proposition for the field extensions $\overline{\Q} \hookrightarrow \C$ and then $\overline{\Q} \hookrightarrow k$ to obtain the desired isomorphism.
	For example, our previous examples for $\textbf{P}_{\C}^n$ hold whenever $\C$ is replaced with $k$ as projective space is proper and integral over a base field.
\end{remark}

\subsection{Positive Characteristic}

So far we have only made progress in the case of characteristic zero.
Certain groups, such as those interested in number theory, may wish to see if we can extend such results
to the positive characteristic case.
Thankfully results are certainly possible, albeit more difficult to formulate.
We will outline one due to Grothendieck, which for certain schemes over a field of characteristic $p \geq 0$ gives us information on the "non-$p$" part of the \'etale fundamental group by relating it to the characteristic 0 case, where one may use results such as the ones previously outlined.\\

\indent We will be interested in a discrete valuation ring $A$, whose field of fractions $K$ is of characteristic 0 but whose residue field $\kappa$ is of characteristic $p \geq 0$.
Following the notation in \cite{Szamuely} let $S$ be the affine scheme $\text{Spec}(A)$, with $\eta \colon \text{Spec}(K) \to S$ and $s \colon: \text{Spec}(\kappa) \to S$ be the generic and closed points of $S$ respectively.
Also fix geometric points $\overline{\eta}$ and $\overline{s}$ of the respective points.

\begin{theorem}[Grothendieck]
	Let $S$ be as above, and let $f: X \to S$ be a proper morphism.
	Fix geometric points $\overline{x}$ and $\overline{y}$ of $X_{\overline{\eta}}$ and $X_s$ respectively.
	\begin{enumerate}
		\item The natural map $\pi_1(X_s, \overline{y}) \to \pi_1(X, \overline{y})$ induced by the map $X_2 \to X$ is an isomorphism.
		\item Assume moreover that $k$ is algebraically closed, $\phi$ is flat, and the geometric fibers $X_{\overline{\eta}}, X_{\overline{s}}$  are reduced.
			Then the natural map $\pi_1(X_{\overline{\eta}}, \overline{x}) \to \pi_1(X, \overline{x})$
	\end{enumerate}
	
\end{theorem}

 \begin{proof}
	 Hard, \todo{cite} reference
 \end{proof}
 
 The main result we have for relating the two contexts depends on a specialization map \todo{Fill in details}

 We first need a lemma


\begin{theorem}[Grothendieck]
	Keep the same notations and assumptions as above.
	Assume that $f$ is proper and smooth with geometrically connected fibers.
	Then the specialization map induces an isomorphism 
	\[\pi_1(X_{\overline{\eta}}, \overline{x})^{p'} \xrightarrow{\sim} \pi_1(X_{\overline{s}}, \overline{y})^{p'}\]
	where the superscripts $(p')$ denote the maximal prime-to-p quotients of the profinite groups involved.
	
\end{theorem}

 \begin{proof}
	 Also long and involved \todo{Try and give a proof?}
 \end{proof}
 

 Now Grothendieck's main result is given below, which we will outline.
\begin{theorem}
	Let $k$ be an algebraically closed field of characterstic $p \geq 0$, and let $X$ be an integral proper normal curve of genus $g$ over $k$.
	For every geometric point $\overline{x}$ of $X$, the group $\pi_1(X, \overline{x})$ has its maximal prime to $p$-quotient $\pi_1(X, \overline{x})^{(p')}$ is isomorphic to the profinite $p'$-compleiton of the group
	\[G = \langle a_1, b_1, \dots, a_g, b_g \,|\, [a_1,b_1], \dots, [a_g, b_g] = 1\rangle\]
	where $[a_i, b_i]$ denotes the commutator $a_i b_i a_i^{-1} b_i^{-1}$.
\end{theorem}

\begin{proof}
	\todo{Give an outline}
\end{proof}



\newpage
\bibliographystyle{plain}
\bibliography{lib}
\end{document}
